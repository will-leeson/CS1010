\documentclass{article}
\usepackage[a4paper, total={6in,9in}]{geometry}
\usepackage{fancyhdr}
\begin{document}
\pagestyle{fancy}
\fancyhead[L]{CS1010 -- Intro to Information Technology}
\fancyhead[R]{Project - Testing a Program}

\section*{Overview}
For this project, you will be testing a program made to play the game Twenty One.
The goal of the game is to have a hand which is closer to the value 21 than the dealer
without going over. If you go over, you lose. This game is played with 4 typical
52 card decks shuffled together. Cards are scored as follows. If the card has a number
on it, its value is that number. If the card has a face on it (Jack, Queen, or King), it
is worth 10 points. The Ace is worth 11 points. The game starts by dealing two cards to 
the player and the dealer. The player will know what cards they have, and they will know
one of the dealers cards. The player will be asked if they want another card. If they say yes,
they will be dealt a card and continue being given the opportunity to take another card until
they decide to stop. If at any point, they go over 21 points, they hand is ended and they lose.

If they don't go over, the dealer will begin accepting cards. The dealer \textbf{must} continue
taking cards until they have 16 or more points. Once they are over, they \textbf{must stop}. 
Once the dealer stops, we compare hands. Whoever is closer to 21 without going over wins. If
players have the same hand, it is a tie.

Players accumulate score overtime, starting with 0. If they win a hand, they get 1 point. 
If they tie, they get 0 points. If they lose, the lose a point. If they get exactly 21, they
get 2 points. Each time a player plays a hand, they will be given the opportunity to play another
or quit. 

Below are 10 things I want you to check are working correctly in the program.
\begin{itemize}
    \item The game ends when a player goes over 21
    \item The dealer stops taking cards once it receives a score of 16 or more
    \item The deck contains the right amount of cards
    \item The deck is shuffled at the start of the game
    \item Dealing a card actually removes it from the deck
    \item Each card is scored correctly
    \item A card is only dealt if the user inputs 'y'
    \item A users score is correctly maintained between hands
    \item Players hands grow when dealt a card
    \item Players receive the correct reward after each hand
\end{itemize}

Of the 10 things, I have intentionally changed the program to make 3 incorrect. For each item, do the following:
\begin{enumerate}
    \item Test the item
    \item If you find the bullet point can be violated
    \begin{enumerate}
        \item Describe how it can be violated (2 pts)
        \item Find what is causing the violation and explain (3 pts)
        \item Fix the violation and describe how you fixed it (4 pts) (If you can't figure out how to fix it, describe how it could be fixed for partial credit)
    \end{enumerate}
    \item If you find the bullet point can't be violated
    \begin{enumerate}
        \item Describe how you tested the program or parts of the program (2 pts)
        \item Explain why you believe this shows the bullet point can't be violated (3 pts)
    \end{enumerate}
\end{enumerate}

When submitting your project, submit a write-up and the code after you've edited it. These should be in two separate files,
a Document file (doc, docx, pdf, etc.) and a Python file (.py). If you do not submit a Python file, you will receive not
credit for 2c (12pts total).

When you test this program, you may find a bug not related to the items listed above. To check if its a bug,
look at the description in the first three paragraphs of this writeup. If you find the program doesn't act
according to this description, you found a bug I didn't know about! If this is the case, I will give at most 5\% extra credit.
To receive the half the extra credit, describe the bug you found (How did you expect the program to behave? How did it behave?).
To receive the rest of the extra credit, implement a fix for the bug.

\end{document}